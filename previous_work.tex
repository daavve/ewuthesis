\chapter{previous work}
\section{Computer Vision}

Computer vision is a very high interest field with a diverse range of applications.  Computer vision primarily has to do with converting a visual input into something the computer can ``understand'' and then perform a specific action based on that observation.  A good example is red-light cameras, if the computer vision system detects what it believes to be someone running the red light, it snaps a picture, records some video as record for a law enforcement officer to evaluate.

Additionally commercial license plate readers have become available.  A license plate scanner takes a still image of a vehicle, then detects the license plate and then ``reads'' the numbers on the license plate of that vehicle.


\section{Optical Character recognition}

Optical Character Recognition is perhapse the most heavily explored area of computer vision.  Specifically, OCR has to do with a computer taking in a digital image, and outputing a text representation of what the computer believes it represents.  OCR systems have been in existance since the 60's and continue to make improvements over time.

The USPS began using OCR systems for automated letter sorting in 19??. 

\section{Document Understanding}

Document Understanding includes OCR, but also includes additional data such as layout and document type.  For example, a Book or Technical Report has a much different layout and organization than a Newspaper or Magazine.  Systems which correctly interpret the organization of a document can produce a more accurate digital representation of said document.  This reduces the labor requirement for digitizing large amounts of printed material, and makes the scanned material more available and discoverable.

\section{Historical Document Understanding}

Computer systems which process historical images face special problems such as degredation of the artifact over time.  There also exists a great variet


\section{Available collections of documents}

Several measeums have digitized thier collections and posted these collections on the Internet.

Digital Public Library:
Metropolitan Measeum of Art:
CADAL \& the million book project:

\section{available collections of Chinese Calligraphy}

\subsection{Metropolitan Meauseum of art history}

This website has digitized and posted online a large portion of thier collection.  The data on this website is broken into works, with a work consisting of several pages.  Each work has an Author, and each page has the digital image.  The Chinese text of each document is also present in Unicode format for each artifact.

\subsection{Seattle Meusium of Art History}, Or something I think.

This webisite consists of an elaborate Flash-bastd website browser.  The images are zoomable to great resolution, and panable.  I do not know of a way to extract image data from inside the Adobe Flash application.  I believe Adobe purposefully makes it difficult to extract the underlying datastreams present in the flash app.
Even so, I'm not sure the art had the text comes in anything other than image format.

\subsection{CADAL Million book Project}

CADAL is in China, it is run by the University of ????.  Cadal is a project funded by the Chinese government with the goal of preserving Chinese historical manuscripts in digital format.  Additionally the project seeks to make these documents available to the world, as to better share Chinese heritage and Culture.

The researchers at CADAL have completed the tedious work of:
1) Identifying the Bounding Box of every character in every page of selected documents (Is this complete?)
2) Identified indifidual characters by thier unicode Chinese script.
3) Additionally, Each character object also holds information about the Author, and the individual work, and page number of the work .


