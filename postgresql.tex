\chapter{using postgresql database}


from archlinux pagte

Note: Commands that should be run as the postgres user are prefixed by [postgres]\$ in this article. You can change to the postgres user by running su - postgres as root. Alternatively, if you use sudo, run sudo -i -u postgres as a regular user.
Before PostgreSQL can function correctly, the database cluster must be initialized:
[postgres]\$ initdb --locale en_US.UTF-8 -E UTF8 -D '/var/lib/postgres/data'
Then start and enable postgresql.service.
Warning: If the database resides on a Btrfs file system, you should consider disabling Copy-on-Write for the directory before creating any database.
Create your first database/user

Tip: If you create a PostgreSQL user with the same name as your Linux username, it allows you to access the PostgreSQL database shell without having to specify a user to login (which makes it quite convenient).
Become the postgres user. Add a new database user using the createuser command:
[postgres]\$ createuser --interactive



Need : local/postgresql-jdbc  for java driver if I want to use pycharm

also:  local/pgmodeler 0.8.2_beta-1, to provide interactive database management,



[root@bigArch ewucal]# su postgres
[postgres@bigArch ewucal]\$ createuser --interactive
Enter name of role to add: dave
Shall the new role be a superuser? (y/n) n
Shall the new role be allowed to create databases? (y/n) y
Shall the new role be allowed to create more new roles? (y/n) n
[postgres@bigArch ewucal]\$ exit

[root@bigArch ewucal]# exit
exit
[dave@bigArch ewucal]\$ createdb calligraphy

also needed to install community/python-psycopg2 to get python to natively connect with PostgreSQL database


Had to edit settings.py to get postgresql database up

DATABASES = {
    'default': {
        'ENGINE': 'django.db.backends.postgresql',
        'NAME': 'calligraphy',
    }
Remember, since I already built a user called dave, which shares the username 'dave' of my current user, then I don't need a username, or password to connect to my database.
