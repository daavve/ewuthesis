\chapter{Calligraphy Validation and Assessment}

Wang Xu is considered ``The great master'' of Chinese Chinese Calligraphy.  Though no known origional examples of his work remain, ancient copies of his work have been auctioned Millions of dollars.

All artifcats so-far discovered containing his work are copies one or more times removed from the origional.

\section{Calligraphy and Imperial China}

Calligraphy became widely practiced during Imperial China.  Individuals hoping to serve in government during that time (when?) had to pass a variety of exams.  Mastership in Calligraphy was considered(by whom?) one of several skills(what?) a person must demonstrate before entering into the Imperial service.  Calligraphy therefore became incorperated into the school system. 

\section{Works of the Great Master}
Accessed:  5 sept 2016
http://www.chinaonlinemuseum.com/calligraphy-chu-suiliang.php 
http://www.chinaonlinemuseum.com/calligraphy-wang-xizhi.php

History:

Wang Xizhi (王羲之) -- Lived 303 to 361
Chinese tradition said no one currently living can be considered a master calligrapher.  Even while Wang Xizhi was alive his work was considered exceptional.  None of the origionals of Wang Xizhi's work currently exist.

The emperor ????? is said to have stutied all contemporary calligraphy works, of these he considered Wang Xizhi's works the best.  Under his riegn the practice of calligraphy was very much invigorated.  Duplicates of Wang Xizhi's work were distributed throughout the empire, and served as exemplars to which students strove.

At during this time all applicants to imperial service had to demonstrate proficiency in calligraphy during the civil examination(Disseration: 69).

(http://ufdc.ufl.edu/UFE0043110/)

Chu Suiliang (褚遂良) -- Lived 596 to 658


Chu chaired a committe go over all works attributed to Wang Xi in the imperial collection.  He produced a catalog listing the works he felt genuin in the official imperial catalog.  The catalog is titled: 
Jin Youjun Wang Xizhi shumu 晋右軍王羲之書目 (List of Calligraphic Works by zhi, General of the Right Army of the Jin dynasty).9 The catalog contains 269 items.(Dissertation)


Wang Xu lived (Where?, When).  He wrote many works of calligraphy and served the Empire as a (What?).  

\section{Emperor who loves Calligraphy}

Emperor took special interest in Wang Xi's works.  He considered Wang Xi's style to be Exemplar, and oversaw the creation and distribution of high-quality copies to schools throughout the empire.

\section{Calligraphy's Rise}

With the Emperor's strong support. The practice of calligraphy grew and flourished.

\section{Calligraphy's Fall}

The Chinese Empire faced significant instability in the form of Wars and civil strife.  During this period several things occured.

\subsection{Loss of Artifcats}

Many(perhaps most) of Wang's origional work became lost or destroyed.

\subsection{Calligraphy in Decline}

With civil society stressed by conflict, imperial focus shifted away from the arts, and more toward warmaking and statescraft..  This neglect permiated the beaurocracy such that even the Caligraphers of the royal court producted works whos qualities ranged from bad to abysmal.  Significantly, the works produced during this time received the same official seals and decoration as the pristene works of the past.

\section{Calligraphy Restored}

During ??? Emperor ??? worked to restore Calligraphy to the place it had during (Previous Emperor who loved Calligraphy's reign)


\section{Our current dillemna}




