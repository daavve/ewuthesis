Abstract, acknowledgements, List of Figures:  These are all roman numerals

Numbered Section:

\section{Introduction}

My project is construction of a web-based online character comparison tool to aid researchs seeking to compare and contrast individual characters, glyphs or letters.  Importantly, characters contains metadata associated with them such as parent documents, authorship, the type of letter, ch




\section{Background}

Digital Library technology comparing individual calligraphy characters within the context of digital libraries largely does not exist.  Calligraphy researchers seeking compare a multitude of individual characters must currently go through the laborious process of flipping between pages of large books, or must use photo management software not well suited to the task.  Photo management software is more geared toward bulding indexable collecitons of work

The work I am doing aims to build upon previous work to the largest extent possible.  Many digital libraries provide access to digital calligraphy collections at no charge.  However the CADAL database was the only set I found which included physical location of individual characters on the page.

Datasets:

1)  The CADAL dataset is a database which contains works from:
    920 seperate authors
    4109 seperate works
    20,026 seperate pages
    110,652 seperate characters
    
    The data is relational in nature, with Authors, Works, pages, and characters all related to each other.

    Many other images or collections of images exist in various libraries.  These image do not have the location information of each character readily available.
    
    
    As stated earlier, my primary goal is to build upon other's work.  The reason behind this is I wish to avoid--to the largest extent possible--the duplication of effort.  Unfortunately, my efforts to attain copies of artifacts such as the Database file that runs CADAL, or the software that powers the website were unsucessfull.
    
    I resorted to a technique called web-scraping.  web-scraping is when an automation technique used to reduce the magnitude of labor required to extract relevant information from a large number of similarly formatted HTML pages.
    
    Websites work by a query and response system, where the web browser sends a request for a specific page, and the web-server responds by sending the appropriate html page.  The page is then rendered and displayed to the user.  The user may then manually enter the viewed data into a database, or manually save the page etc.
    
    In web-scraping, a computer program called a scraper replaces the browser.  From the server perspective the scraper behaves similarly to a brower.   The server receives requests from the scraper, and server html pages in response.  My scraper fetches thousands of seperate HTML pages, and saves these pages to disk for later processing.
    
    
    
    
    
    

1)  CADAL, this system allows individual characters to be searched and then viewed within the context of source documents.  The system has a public-facing web frontend which enables anyone with an internet connection and a browser to view the collection.  The dataset is read-only by nature so can only be used for reference.  The data is extensive, it exists as a collection of HTML pages, and images.  The system gives two modes, one the viewing characters through queries, and another for browsing 

2)  A dataset exists which includes scanned calligraphic works with information about the characters included as metadata.  This dataset focuses only on calligraphy, and I would have liked to use it to provide me multiple seperate data sources. I was unable to find the dataset online, the messages I sent to the authors were not returned.






2.1     Memory Corruption Attacks
2.2     System Secutiry
3       Current Sandbox Implementaitons
3.1     Unix Based Implementations
3.2     Windows Based Implementations
3.3     Developer API Implementations
4       Sandbox Technical Overview
5       Project Details
5.1     Project Design
5.2     Framework Outline
5.3     Hooking System Calls
5.4     Fetching Parameters
5.5     Framework Usage
6       Proof of Concept
6.1     Validate Process Creation
6.2     DEP Bypass Prevention
7       Future Work
8       Conclusion

Unnumbered Pages: References, Vita

