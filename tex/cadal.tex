


The work I am doing aims to build upon previous work to the largest extent possible.  Many digital libraries provide access to digital calligraphy collections at no charge.  However the CADAL database was the only set I found which included physical location of individual characters on the page.

Datasets:

1)  The CADAL dataset is a database which contains works from:
    920 seperate authors
    4109 seperate works
    20,026 seperate pages
    110,652 seperate characters
    
    The data is relational in nature, with Authors, Works, pages, and characters all related to each other.

    Many other images or collections of images exist in various libraries.  These image do not have the location information of each character readily available.
    
    
    As stated earlier, my primary goal is to build upon other's work.  The reason behind this is I wish to avoid--to the largest extent possible--the duplication of effort.  Unfortunately, my efforts to attain copies of artifacts such as the Database file that runs CADAL, or the software that powers the website were unsucessfull.
    
    I resorted to a technique called web-scraping.  web-scraping is when an automation technique used to reduce the magnitude of labor required to extract relevant information from a large number of similarly formatted HTML pages.
    
    Websites work by a query and response system, where the web browser sends a request for a specific page, and the web-server responds by sending the appropriate html page.  The page is then rendered and displayed to the user.  The user may then manually enter the viewed data into a database, or manually save the page etc.
    
    In web-scraping, a computer program called a scraper replaces the browser.  From the server perspective the scraper behaves similarly to a brower.   The server receives requests from the scraper, and server html pages in response.  My scraper fetches thousands of seperate HTML pages, and saves these pages to disk for later processing.
    
    
    
    
    
    

1)  CADAL, this system allows individual characters to be searched and then viewed within the context of source documents.  The system has a public-facing web frontend which enables anyone with an internet connection and a browser to view the collection.  The dataset is read-only by nature so can only be used for reference.  The data is extensive, it exists as a collection of HTML pages, and images.  The system gives two modes, one the viewing characters through queries, and another for browsing 

2)  A dataset exists which includes scanned calligraphic works with information about the characters included as metadata.  This dataset focuses only on calligraphy, and I would have liked to use it to provide me multiple seperate data sources. I was unable to find the dataset online, the messages I sent to the authors were not returned.


1)  The Cadal Calligraphy website offers 3 distinct browsing modes for viewing the calligraphy collection:
    (book mode)
        In this mode users may browse a variety of books.  These books are written in Chinese Calligraphy.  The books themselves consist a collection of scanned images.  Each image represents a seperate page of the book.  The books are viewed through an interactive Adobe flash web application.
        
        The application gives the user a set of interactive buttons which allow the user to browse between pages, zoom-in/out etc.  
        
        Unfortunately, my account is a probationary account, where I am not permitted to view any pages past page 10 using the flash browser.
        If you click the info button you get redirected to a page containing a chaper index of the book as well as relevant information about the book you may edit.

    (worklist mode)
        In this mode users may search and browse collections of different works.
            A work is defined as one or more pages representing a single piece of work attributed to one calligrapher.
            Frequently (but not usually) the works include a block of text in a window at the bottom.  If present, all pages of the translated work are printed at the bottom of the screen in a messagebox.
            
    (character mode)
        In this mode users may browse, search for or view individual characters.  The characters are printed in a table, with up to 18 characters visible on the page at any given time.  The following search fields are searchable:
            * character : 检索字(词)
            * calligrapher :  书家
            * dynasty :  朝代
            * Chriography :  书体 , a dropdown menu consisting of 6 different styles, all: 所有 or (楷书) (行书) (篆书) (隶书) (草书) (其它)
            
            An additional drop-down menu reveals additional search options.
            
            * 偏旁笔画 : Radical strokes, a dropdown consisting of 1, 2, 3......15 strokes
            *  偏旁 : Radical, a dropdown consisting of between 1 and 
            
            
            These additional search options all have to do with different characteristics of a single character.  example

