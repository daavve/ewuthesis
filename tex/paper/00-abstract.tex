ABSTRACT\\

The character classification problem has not been solved in the general case.  Character classification can be done if the characters are restricted to certain fonts, but not for unconstrained fonts and especially not for free-flowing or badly degraded images.

Current progress in character and object classification depends on the existance of large labeled data sets from which a classification model may be learned.

The most comprehensive dataset of ancient chinese calligraphy is CADAL

I made a copy of this dataset and it's associated images

The quality of models directly related to the quality of the dataset used to train such models.

I performed a visual audit of the dataset and made some improvement on said set,

A classification model built from that dataset was able to improve character detection rates over an untrained model.

Heart of issue:  CADAL Dataset bounding box:  Improving the accuracy and completeness of the bounding boxes of the CADAL Dataset produces better quality classifiers.


Areas still needing done:  Character Name,  Cannot reliably put in Character name because I cannot read chinese.

Visual verification of bounding boxes of characters within CADAL dataset and impacts in infered models.


Primary contribution of my work is the that the core results of my work are AVAILABLE, and VERIFIABLE

Primary contribution:  My work builds directly on already existing publicably available work.  That my results are available for others to fact check.

Hypothesis:  A classifier can be trained using the CADAL dataset to improve identification of characters from the "Collected Characters" Stelle.

Answer to hypothesis:  The classifier trained from the CADAL dataset performs poorly on the Stelle.  This poor showing is caused by incomplete and inaccurate data inside the CADAL dataset.  Once the CADAL dataset is corrected to fill in missing data and correct inaccurate data the resulting classifier performs much well on the Collected characters Stelle.

\newpage
