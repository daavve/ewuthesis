\documentclass{ewuthesis}
\usepackage{hyperref}
\hypersetup{pdftitle={CaliSet: The Calligraphy Dataset}}
\usepackage{morefloats}
\usepackage{graphbox}
\graphicspath{ {images/} }
\usepackage[toc, xindy]{glossaries}

\newglossaryentry{cadal}{name={cadal},description={China Academic Digital Associate Library}}
% \newacronym{cadal}{cadal}{CADALChina Academic Digital Associate Library}

\makeglossaries

\begin{document}
    \title{CaliSet: The Calligraphy Dataset}
    \author{David A. McInnis}    
    \degreeyear{2016}
    \degreeterm{Fall}         %Winter, Spring, Summer, Fall
    \degree{Master of Science}
    \department{Computer Science} 
    
    \committeefirst{Kosuke Imamura, PhD}
    \committeesecond{Carol Taylor, PhD}
    \committeethird{Esteban Rodriguez-Marek, M.S.}
    
    \birthplace{Spokane, Washington}
    \associatedegree{Associate of Applied Science}
    \associateyear{2007}
    \associateschool{Community College of the Air Force}
    \bachelorsdegree{Bachelors of Science}
    \bachelorsyear{2010}
    \bachelorsschool{Embry-Riddle Aeronautical University}
    
    
    \frontmatter
    \maketitle
    \makesigpage
    \makelibrarystatement{}
    \subimport{}{00-abstract.tex}
    \subimport{}{01-acknowledgements.tex}
    \mainmatter
    \tableofcontents
    \listoffigures{}
    \subimport{}{02-introduction.tex}
    \subimport{}{03-background.tex}
    \subimport{}{04-existing-collections.tex}
    \subimport{}{05-web-scraping.tex}
    \subimport{}{06-overview-of-cadal-data}
    \subimport{}{07-website-construction.tex}
    \subimport{}{08-website-evaluation.tex}
    \subimport{}{09-future-work.tex}
    here is \gls{cadal}
    \chapter{glossary}

% \bibliographystyle{acm}

% wazzup\cite{Gao:2012:CDC:2232817.2232889} fellas

% \bibliography{cadal}


    \backmatter
   

    
    \makevita
    
    
\end{document}
