\chapter{existing collections}
Calliset first required an image catalog, along with a corresponding database of character name, attributions, and positions in order to be effective.  In order to reduce the burden on my historian, I first searched the available research and image datasets.  This way I could aviod both unneccessary work and duplication of effort.




\section{private collections}

\subsection{Artifacts}

\subsection{Books}

\section{World Digital library Project}

\subsection{Metropolitan Meausium of Art}

\section{China Million Book Project}

\subsection{CADAL Digital Library}

\subsubsection{Cadal Calligraphic Database}

\section{Available collections of documents}

Several measeums have digitized thier collections and posted these collections on the Internet.

Digital Public Library:
Metropolitan Measeum of Art:
CADAL \& the million book project:

\section{available collections of Chinese Calligraphy}

\subsection{Metropolitan Meauseum of art history}

This website has digitized and posted online a large portion of thier collection.  The data on this website is broken into works, with a work consisting of several pages.  Each work has an Author, and each page has the digital image.  The Chinese text of each document is also present in Unicode format for each artifact.

\subsection{Seattle Meusium of Art History}, Or something I think.

This webisite consists of an elaborate Flash-bastd website browser.  The images are zoomable to great resolution, and panable.  I do not know of a way to extract image data from inside the Adobe Flash application.  I believe Adobe purposefully makes it difficult to extract the underlying datastreams present in the flash app.
Even so, I'm not sure the art had the text comes in anything other than image format.

\subsection{CADAL Million book Project}

CADAL is in China, it is run by the University of ????.  Cadal is a project funded by the Chinese government with the goal of preserving Chinese historical manuscripts in digital format.  Additionally the project seeks to make these documents available to the world, as to better share Chinese heritage and Culture.

The researchers at CADAL have completed the tedious work of:
1) Identifying the Bounding Box of every character in every page of selected documents (Is this complete?)
2) Identified indifidual characters by thier unicode Chinese script.
3) Additionally, Each character object also holds information about the Author, and the individual work, and page number of the work .








Image Datasets are used extensively in both Research and Industry to push the frontiers of machine learning.  Searchable images also help e-commerce companies such as <newcite> find visually similar styles for customers.

Today thousands of image datasets exist containing millions of individual photos available online to researchers and industry.  <newcite>.  These datasets typically consist of images and associated metadata(usually text) annotating the details of the picture in a structured way that computer can readily interpret.  An important purpose of these datasets is to permit is to reduce duplication of effort, as well improve reproducability of research.  While a great many datasets exist for scanned manuscripts.  Relatively few of these contain chinese calligraphy.

The places that you can find collections of chinese calligraphy documents online:  Meuseum online collections such as the metropolitan meusium of art.  Bulk archiving sites such as archive.org.  The information from collections tends to be relatively unstructured and incomplete from a data processing standpoint.  This is the result of such online collections being optimized for interactive browsing and not bulk analysis.

A great many datasets of written chinese text exist, but I found only two which contain Chinese calligraphy.  Of these two sets, I was only able to gain access to one, the CADAL Calligraphy database.  This Calligraphy contains importantly, images of works, combined with meta-information indicating the position of each character on the page, as along with other pertinant information.  This metadata is critical, because machines cannot readily interpret the calligraphy images.  Additionally, many of the texts are very old, and the language has changed enought that even native speakers may have trouble desciphering the script.

The work I am doing aims to build upon previous work to the largest extent possible.  Many digital libraries provide access to digital calligraphy collections at no charge.  However the CADAL database was the only set I found which included physical location of individual characters on the page.

   Many other images or collections of images exist in various libraries.  These image do not have the location information of each character readily available.

Datasets:

1)  The CADAL dataset is a database which contains works from:
    920 seperate authors
    4109 seperate works
    20,026 seperate pages
    110,652 seperate characters


Unfortunately, my access to this calligraphy database was mediated by the CADAL website.  This website presented a rich variety of data about individual characters, such as location within parent work.  Like the Meusium collections mentioned earlier, the CADAL images and Database were very much curated by a web front-end.  CADAL provides no option export data into a computer-readable form.  My requests for one remain unanswered.

    Why CADAL website does not meet the needs of my project:
        1)  Work browsing interface does not allow for easy selection of an individual character from an existing work.
        2)  The character search page is limited to only the first 18 results at any time.
            *The character images themselves are generally low resolution, and highly artifact-ed, with some characters exhibiting this trait more profoundly than others.
        3)  When the source page, with the bounding box is presented, overlay-ed in the source document, you get the following issues:
            *  Source page is low resolution
            *  The bounding box is quite thick and frequently obscures edges of the character
            *  Displayed page is quite small, and the interface frustrates efforts to enlarge, or print them.
        4)  High resolution images of pages exist on the CADAL server
            * But only reduced resolution versions of these images are accessible when browsing the website.
                + These available images suffer from the following quality issues:
                    -Critical details are fuzzed, during down-sampling
                    -Lossy JPEG compression further degrades image by smoothing textures, and introducing artifacts such as ringing
        5)  The CADAL website exists in mainland China.  Any download of large files(Files larger than about 500Kb), are severely slowed by the Great firewall of china.
            *Many of the image files were larger than 3Mb, which means they could take many minutes of waiting to download and view
        
        
    It might have been possible to write a web gateway, which provides intermediate access to CADAL.  This could have overcome the image resolution problem, but could not easily solve the incomplete query problem.
    
    To summarize, I faced 3 problems when accessing the CADAL website directly:
        *  Low quality images
        *  Incomplete query results for characters
        *  Bounding boxes which frequently obstructed portions of characters.
                    

I used a technique called web-scraping to make an indirect copy of the CADAL Character and Works database.  With this copy I am able to answer the above questions.  Additionally, since I now posses a copy the whole database, I can now make the database available in a form that other computer researchers can use directly.


   I resorted to a technique called web-scraping.  web-scraping is when an automation technique used to reduce the magnitude of labor required to extract relevant information from a large number of similarly formatted HTML pages.






