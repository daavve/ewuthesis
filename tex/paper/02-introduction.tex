\chapter{Introduction}

Historians often face the daunting task of having to organize and assess vast numbers of documents.  This problemm is especially acute with Chinese History.  This is because written records of chinese history go back thousands of years.  Most are not well kept, well organized, well proctored, or well preserved.

This leads to a confusing and difficult mass of texts.  Add to this the prospect that certain texts are considered quite valuable and counterfeiters will go to great lengths to decieve collectors and historians into mistakenly certifying fratulent works as authentic.

Additionally, many origional works no longer exist.  Since copying technology was an artisan skill 

My work presents Calliset(Calligraphy Dataset).  THe purpose of Calliset is the following:

(1)  To maintain and host a duplicate copy of the CADAL Calligraphy Character Dataset in North America.  The researchers in China put prodigous effort into producing this data for the Scientific community\cite{Zhang:2011}.  While CADAL has made efforts to ensure the durability of it's scanned images through partnership with archive.org.  Organization of my website enables automatic indexing and archiving.


My work consists of the following purposes:
The first is to assist historians who wish to quickly and easily compare the details of two or more characters attributed to the same author.
The second is to present the artifacts of my research in a way that permits easy duplication and validation of my work.  In this way future researchers, and the public can use and adapt my work to better serve their individual needs.

Archival software exists to relive some of the administrative burden from Historians.  However, the software's primary focus is on relieving organizational difficulties(is it?)  

My current client is facing an issue where he/she has a large number of books,  These books contain printed characters each one attributed to a single work from a single author.  he/she must perform the following steps when comparing characters:

1)  Open a book of the author he-she is interested.  Flip through the book, one page at a time, searching for a similar character, when that character if found then, place each book by each other, and compare the two characters visually.

2) Inportant notes about calligraphy books.  1)  They tend to not have indexes, so searching page-by-page, or building your own index becomes necessary.  2)  They frequently do not have characters typed, relying instead on the reader to descipher the (sometimes very poor quality) copy.

    
Why the website?
    * Many historians are not computer savvy, so an interface they are already familiar with may feel more natural to them.
    * Relieves the burden of having to download, install, configure, and debug experimental software.
    * Accessible to anyone who uses the internet.
    * Provide  a platform to which many people can use and contribute to.

