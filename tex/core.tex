

"The problem of distinguishing copies from originals has become ubiquitous in the study of Chinese Calligraphy" (79: Dissertation)

    Question:  Calligraphy researchers seeking to compare and contrast that same character by different works must perform arduous and tedious work.
                Kosuke presented the image of many different books or stacks of paper, with the researcher flipping between them,  maybe using a magnifying glass to compare and contrast each of the characters.  A large portion of the researchers time becomes wasted in scanning the physical pages over and over again.


    Answer:  I propose a system which makes checking related characters intuitive and fast.  I present CCT(Character Comparison Tool).  CCT provides the researcher with a high quality zoomable and navigatable image of the work.  Overlayed on this image are bounding boxes representing individual characters in the database.  Selecting any of these boxes brings up a carrousel of similar images at the bottom of the page.  Selecting any of these characters brings up the source document for that character in a new window, with source character already selected(todo).  

Motivation:
    *  A strong market exists for calligraphic works.  Works certified as authentic(nearly original) can command high prices, which creates an incentive for forgeries.  A lack of good records, loss of original artifacts, and the inconsistencies in duplication and distribution make the certification challenge extremely difficult.
    
Why the website?
    * Many historians are not computer savy, so an interface they are already familiar with may feel more natural to them.
    * Relieves the burden of having to download, install, configure, and debug experimental software.
    * Accessible to anyone who uses the internet.
    * Provide  a platform to which many people can use and contribute to.

    
Previous Work:
    1) Digital Libraries.  Online repositories of information provided over the internet.
        *World digital library
        *China Million Book Project
            *CADAL:
                -CADAL Calligraphic database
    2) ACADEMIC work:
        *Handwriting analysis
            +signature verification
            

What I found:
    1)  There exist two databases which have positional information of Chinese calligraphy.
    2)  CADAL, and Other guys < never was able to get in touch with other guys >
        *CADAL calligraphy page contains thousands of pages of scanned text, as well as over 100K Chinese characters as well as associated metadata.

    Cadal has many usefull features that could fit my project:
        *)  The cadal website has the ability to search for individual characters by submitting a variety of queries.
            +)  By character
            +)  By author
            +)  By dynesty, etc.


    Why CADAL does not meet the needs of my project:
        1)  Work browsing interface does not allow for easy selection of an individual character from an existing work.
        2)  The character search page is limited to only the first 18 results at any time.
            *The character images themselves are generally low resolution, and highly artifacted, with some characters exibhiting this trait more profoundly than others.
        3)  When the source page, with the bounding box is presented, overlayed in the source document, you get the following issues:
            *  Source page is low resolution
            *  The bounding box is quite thick and frequently obscures edges of the character
            *  Displayed page is quite small, and the interface fustrates efforts to enlarge, or print them.
        4)  High resolution images of pages exist on the CADAL server
            * But only reduced resolution versions of these images are accessable when browsing the website.
                + These available images suffer from the following quality issues:
                    -Critical details are fuzzed, during downsampling
                    -Lossy JPEG compression further degrades image by smoothing textures, and introducing artifacts such as ringing
        5)  The CADAL website exists in mainland China.  Any download of large files(Files larger than about 500Kb), are severely slowed by the Great firewall of china.
            *Many of the image files were larger than 3Mb, which means they could take many minutes of waiting to download and view
        
        
    It might have been possible to write a web gateway, which provides intermediate access to CADAL.  This could have overcome the image resolution problem, but could not easily solve the incomplete query problem.
    
    To summarize, I faced 3 problems when accessing the CADAL website directly:
        *  Low quality images
        *  Incomplete query results for characters
        *  Bounding boxes which frequently obstructed portions of characters.
                    



    In order to product my own website I needed to acquire the following things from CADAL:
        1)  Character information with the following:
            Mark:
            Author:
            Image:
            Image of Parent document:
            Locational position of character in parent document.
        
        2)  Image information with the following:
            Author,
            Work info:
                Page of work
                Parent document
            
            The following was sometimes available:
                Name of Work,
                Transcript of work
                
        
            














    Unexpected problems that violated my initial assumptions:
        * not all calligraphic works have a text script accompanying them
        * not all calligraphic works with text have individual characters mapped
        * of those calligraphic works with characters mapped
            -pages only have a portion of characters mapped.
            -Quality of mapped characters is generally quite good except:
                Bounding boxes frequently overlap
                Bounding boxes frequently cut off portions of the character.

What I was able to do:
    My findings:
        * There exists enough work to compare many dozen of the same characters, from and within works of the same author, on a character by character basis. <This is already known>
        *  Many of the characters encoded in CADAL Calligraphy many characters are single specimens and therefore defy individual comparisons.  <also already known>  The Chinese language contains a vast variety of characters, many of which are used only rarely.
    
    My results:
        *  I created a website that premits browsing uploaded images of calligraphy texts.
            +Bounding boxes are overlayed on top of characters on the 
                *These bounding boxes follow characters during navigation (Zoom and drag)
            +To the best of my knowledge, this is the first application to allow easy and direct navigation between similar characters on scanned high resolution, zoomable images.


    My future work:
        *  Preform user surveys and usability testing to measure the improvement (if any) over existing methods
        *  Create an interface which enables website users to modify existing data on the site.
        *  Use a learning algorithm to "guess" the position of un-annotated characters.
        *  OCR performance on calligraphic characters is abysmal.  Attack this problem
        *  Use a comparison algorithm to "measure" and perhaps draw attention to important similarities / differences between characters.




