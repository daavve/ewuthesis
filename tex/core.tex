王羲之

王羲之集字圣教序 : Wang set the holy word sequence

*or*

怀仁集王羲之 : Wai Yan Wang set

*or*

七佛圣教序 : Chilbulbong Sacred Order  http://sns.91ddcc.com/t/63381

*or*

雁塔圣教序: Yanta holy church order

*recommended by biadu*

王羲之圣教序墨迹本: Wang Xizhi holy church order of the ink

*or*

多宝塔碑

Accessed:  5 sept 2016
http://www.chinaonlinemuseum.com/calligraphy-chu-suiliang.php 
http://www.chinaonlinemuseum.com/calligraphy-wang-xizhi.php


History:

Wang Xizhi (王羲之) -- Lived 303 to 361

During his life he was regarded as an excelent Calligrapher.
No


Chu Suiliang (褚遂良) -- Lived 596 to 658


Chu chaired a committe go over all works proported to Wang Xi in the imperial collection.  He produced a catalog listing the works he felt genuin. in the official imperial catalog.  The catalog is titled: 
Jin Youjun Wang Xizhi shumu 晋右軍王羲之書目 (List of Calligraphic Works by zhi, General of the Right Army of the Jin dynasty).9 The catalog contains 269 items.(Dissertation)

Taizong promoted Wang Xizhi's style enthusiastically.  The emperor had studied 
(http://ufdc.ufl.edu/UFE0043110/)







*Big wild Goose Pagoda in Xi'an of Shaanxi Province* 大雁塔
Kept under the pagoda is a stone tablet with an inscription made by Chu Suiliang, a calligrapher in the Tang Dynasty, which is an important relic. (http://www.china.org.cn/english/TR-e/43279.htm)

Base of the pagoda
Preserved on the four stone doors in the base of the pagoda are exquisite engravings of the Tang. Two steles with the Preface to the Sacred Religion written by the famous Tang calligrapher Chu Suiliang are set into the walls on the either side of the south door of the pagoda. (http://www.travman.com.au/cities/City_Xian_1.htm)

Sheng jiao xu
 Emperor Tai zong wrote "An Introduction to the Sacred Teaching of Monk Tripitaka of the Great Tang Dynasty", followed by Crown Prince Li Zhi's "Notes on the Introduction to the Sacred Teachings of Monk Tripitaka of the Great Tang Dynasty". Chu Suiliang, a famous calligrapher of the Tang Dynasty, (http://www.chinatravelrus.com/cityguide/xian/big-wild-goose-pagoda.html)

《后唯识记》(http://web.cs.iastate.edu/~jia/album/2005/album-xian-tablets.html)



I to actually have it in CADAL:

http://127.0.0.1:8000/work/3523

http://127.0.0.1:8000/auth/141

褚遂良
https://en.wikipedia.org/wiki/Chu_Suiliang



3)  Character comparison (Kosuke's Task):
    Question:   Given a Calligraphic work <Kosuke's book> featuring a collection of characters hand-copied from a variety of parent works.
                Given that none of the originals exist, and that both Kosuke's book as well as the parent works we have available are hand copied one or more times from the originals.
                For any given Character in Kosuke's book, can we determine source document and specific character it was copied from?
                
Why people might find this important:
    *  The character's in Kosuke's are considered the "best" examples of calligraphy by Wang-Xu.
    *  Understanding which source documents contributed to this book would yield a deeper understanding of which scripts contributed most strongly to this collection of exemplar characters.
    *  A strong market exists for calligraphic works.  Works certified as authentic(nearly original) can command high prices, which creates an incentive for forgeries.  A lack of good records, loss of original artifacts, and the inconsistencies in duplication and distribution make the certification challenge extremely difficult.
    
Previous Work:
    1) Digital Libraries.  Online repositories of information provided over the internet.
        *World digital library
        *China Million Book Project
            *CADAL:
                -CADAL Calligraphic database
    2) ACADEMIC work:
        *Handwriting analysis
            +signature verification
            

Process:  What I had to do:
    1)  Get together a collection of characters origionally written by 王羲之.
    2)  Scan Kosuke's book
    3)  Organize the characters in Kosuke's book as well CADAL characters found by other means in a consistent format
    
        Build a tool which allows the user to view individual calligraphic pages, characters, bounding boxes for characters, all displayed in a single page.
    
What I found:
    1)  There exist two databases which have segmented characters written by 王羲之.
    2)  CADAL, and Other guys < never was able to get in touch with other guys >
        *CADAL calligraphy page contains thousands of pages of scanned text, as well as over 100K Chinese characters as well as associated metadata.
    Unexpected problems that violated my initial assumptions:
        * not all calligraphic works have a text script accompanying them
        * not all calligraphic works with text have individual characters mapped
        * of those calligraphic works with characters mapped
            -pages only have a portion of characters mapped.
            -Quality of mapped characters is inconsistent
                * many characters have different boarders in image than attached coordinates
                * Bounding boxes frequently overlap, and sometimes cut off important parts of a character.

What I was able to do:
    My findings:
        * There exists enough work to compare many dozen of the same characters, but from different works by the same author, on a character by character basis. <This is already known>
        *  Of the characters encoded in CADAL Calligraphy many characters are single specimens and therefore defy individual comparisons.  <also already known>
    
    My results:
        *  I have failed in my attempt to reliably identify any of the source characters in Kosuke's book.
        
        *  I have gained useful insight into the problem though.
            +I transformed data from CADAL into a format which serves my purpose.
            +I transformed the CADAL data in JSON, a general-purpose format that other users can use for analysis.
            
    My future work:
        *  Make more progress in my calligraphy website.
        *  Enable crowd-sourcing of segmentation and attribution data, as well as validation of said data


Why the website?
    * Accessible to anyone who uses the internet.
    * Provide  a platform to which many people can contribute to.
    * Crowd-sourcing opportunities for segmentation work.
        +Why we need crowd-sourcing? <number of unsegmented results, humans still better at this visual task>


