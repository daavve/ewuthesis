

"The problem of distinguishing copies from originals has become ubiquitous in the study of Chinese Calligraphy" (79: Dissertation)

    Question:  Calligraphy researchers seeking to compare and contrast that same character by different works must perform arduous and tedious work.
                Kosuke presented the image of many different books or stacks of paper, with the researcher flipping between them,  maybe using a magnifying glass to compare and contrast each of the characters.  A large portion of the researchers time becomes wasted in scanning the physical pages over and over again.


    Answer:  I propose a system which makes checking related characters intuitive and fast.  I present CCT(Character Comparison Tool).  CCT provides the researcher with a high quality zoomable and navigatable image of the work.  Overlayed on this image are bounding boxes representing individual characters in the database.  Selecting any of these boxes brings up a carrousel of similar images at the bottom of the page.  Selecting any of these characters brings up the source document for that character in a new window, with source character already selected(todo).  

3
    *  A strong market exists for calligraphic works.  Works certified as authentic(nearly original) can command high prices, which creates an incentive for forgeries.  A lack of good records, loss of original artifacts, and the inconsistencies in duplication and distribution make the certification challenge extremely difficult.
    
Previous Work:
    1) Digital Libraries.  Online repositories of information provided over the internet.
        *World digital library
        *China Million Book Project
            *CADAL:
                -CADAL Calligraphic database
    2) ACADEMIC work:
        *Handwriting analysis
            +signature verification
            

What I found:
    1)  There exist two databases which have segmented characters written by 王羲之.
    2)  CADAL, and Other guys < never was able to get in touch with other guys >
        *CADAL calligraphy page contains thousands of pages of scanned text, as well as over 100K Chinese characters as well as associated metadata.
    Unexpected problems that violated my initial assumptions:
        * not all calligraphic works have a text script accompanying them
        * not all calligraphic works with text have individual characters mapped
        * of those calligraphic works with characters mapped
            -pages only have a portion of characters mapped.
            -Quality of mapped characters is inconsistent
                * many characters have different boarders in image than attached coordinates
                * Bounding boxes frequently overlap, and sometimes cut off important parts of a character.

What I was able to do:
    My findings:
        * There exists enough work to compare many dozen of the same characters, but from different works by the same author, on a character by character basis. <This is already known>
        *  Of the characters encoded in CADAL Calligraphy many characters are single specimens and therefore defy individual comparisons.  <also already known>
    
    My results:
        *  I created a website that premits browsing uploaded images of calligraphy texts.
            +Bounding boxes are overlayed on top of characters on the 
                *These bounding boxes follow the 
            
    My future work:
        *  Preform user surveys and usability testing to measure the improvement (if any) over existing methods
        *  Create an interface which enables website users to modify existing data on the site.
        *  Use a learning algorithm to "guess" the position of un-annotated characters.
        *  OCR performance on calligraphic characters is abysmal.  Attack this problem
        *  Use a comparison algorithm to "measure" and perhaps draw attention to important similarities / differences between characters.


Why the website?
    * Accessible to anyone who uses the internet.
    * Provide  a platform to which many people can contribute to.
    * Does not require the user to install specific software.



