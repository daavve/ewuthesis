
1)  The Cadal Calligraphy website offers 3 distinct browsing modes for viewing the calligraphy collection:
    (book mode)
        In this mode users may browse a variety of books.  These books are written in Chinese Calligraphy.  The books themselves consist a collection of scanned images.  Each image represents a seperate page of the book.  The books are viewed through an interactive Adobe flash web application.
        
        The application gives the user a set of interactive buttons which allow the user to browse between pages, zoom-in/out etc.  
        
        Unfortunately, my account is a probationary account, where I am not permitted to view any pages past page 10 using the flash browser.
        If you click the info button you get redirected to a page containing a chaper index of the book as well as relevant information about the book you may edit.

    (worklist mode)
        In this mode users may search and browse collections of different works.
            A work is defined as one or more pages representing a single piece of work attributed to one calligrapher.
            Frequently (but not usually) the works include a block of text in a window at the bottom.  If present, all pages of the translated work are printed at the bottom of the screen in a messagebox.
            
    (character mode)
        In this mode users may browse, search for or view individual characters.  The characters are printed in a table, with up to 18 characters visible on the page at any given time.  The following search fields are searchable:
            * character : 检索字(词)
            * calligrapher :  书家
            * dynasty :  朝代
            * Chriography :  书体 , a dropdown menu consisting of 6 different styles, all: 所有 or (楷书) (行书) (篆书) (隶书) (草书) (其它)
            
            An additional drop-down menu reveals additional search options.
            
            * 偏旁笔画 : Radical strokes, a dropdown consisting of 1, 2, 3......15 strokes
            *  偏旁 : Radical, a dropdown consisting of between 1 and 
            
            
            These additional search options all have to do with different characteristics of a single character.  example
        
         
            
            
