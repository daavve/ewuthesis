*Big wild Goose Pagoda in Xi'an of Shaanxi Province* 大雁塔
Kept under the pagoda is a stone tablet with an inscription made by Chu Suiliang, a calligrapher in the Tang Dynasty, which is an important relic. (http://www.china.org.cn/english/TR-e/43279.htm)

Base of the pagoda
Preserved on the four stone doors in the base of the pagoda are exquisite engravings of the Tang. Two steles with the Preface to the Sacred Religion written by the famous Tang calligrapher Chu Suiliang are set into the walls on the either side of the south door of the pagoda. (http://www.travman.com.au/cities/City_Xian_1.htm)

Sheng jiao xu
 Emperor Tai zong wrote "An Introduction to the Sacred Teaching of Monk Tripitaka of the Great Tang Dynasty", followed by Crown Prince Li Zhi's "Notes on the Introduction to the Sacred Teachings of Monk Tripitaka of the Great Tang Dynasty". Chu Suiliang, a famous calligrapher of the Tang Dynasty, (http://www.chinatravelrus.com/cityguide/xian/big-wild-goose-pagoda.html)

《后唯识记》(http://web.cs.iastate.edu/~jia/album/2005/album-xian-tablets.html)



I to actually have it in CADAL:

http://127.0.0.1:8000/work/3523

http://127.0.0.1:8000/auth/141

褚遂良
https://en.wikipedia.org/wiki/Chu_Suiliang


王羲之集字圣教序 : Wang set the holy word sequence

*or*

怀仁集王羲之 : Wai Yan Wang set

*or*

七佛圣教序 : Chilbulbong Sacred Order  http://sns.91ddcc.com/t/63381

*or*

雁塔圣教序: Yanta holy church order

*recommended by biadu*

王羲之圣教序墨迹本: Wang Xizhi holy church order of the ink

*or*

多宝塔碑: Pagoda monument

*or*

兰亭序: Preface

图片: Photos

)  Character comparison (Kosuke's Task):
    Question:   Given a Calligraphic work <Kosuke's book> featuring a collection of characters hand-copied from a variety of parent works.
                Given that none of the originals exist, and that both Kosuke's book as well as the parent works we have available are hand copied one or more times from the originals.
                For any given Character in Kosuke's book, can we determine source document and specific character it was copied from?
                
Why people might find this important:
    *  The character's in Kosuke's are considered the "best" examples of calligraphy by Wang-Xu.
    *  Understanding which source documents contributed to this book would yield a deeper understanding of which scripts contributed most strongly to this collection of exemplar characters.

Process:  What I had to do:
    1)  Get together a collection of characters origionally written by 王羲之.
    2)  Scan Kosuke's book
    3)  Organize the characters in Kosuke's book as well CADAL characters found by other means in a consistent format
    
        Build a tool which allows the user to view individual calligraphic pages, characters, bounding boxes for characters, all displayed in a single page.
