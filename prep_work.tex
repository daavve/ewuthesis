\chapter{Prep Work}

\section{CADAL Library}

The Chinese Academic Digital Associative Library (CADAL) is an effort by the Chinese government to digitize historic chinese works.  It is part of the Million Book project, which is an international effort to make historic written works worldwide universally available through the internet.

\subsection{CADAL Calligraphy Database}
The CADAL Calligraphy Database is a sub-project within CADAL.  It consists of specific calligraphic works, organized and cataloged:


The fetched data is not uniform in annotation.  For example.  Not every work is assigned an author, and not every work includes a transcript.  also.  I only retrieved pages associated with either a specific work and/or character.

Aditionally the CADAL website offers no method of easily importing / exporting the data.  

\section{aquiring Character Data}

The CADAL Server in china delivers images at wildly inconsistant speeds.  Sometimes as slow at 20KB/s sometimes as fast at 2.5MB/s.  But usally above 50KB/s and below 500KB/s.  Slow bulk data transfer is a notorious problem when transfeing information in or out of China.  It took over 5 days to retrieve about 40Gb of images and store it on my Server.

The.  The researches at Cadal have also translated individual characters, and annotated Authorship, what what work that character belongs in.

Eventually I discovered that when I sent a particular HTTP request to the web server, I could get a HTML table back of 18 characters,  then is was simply a matter of asking the webserver for the next page and the next page until I received an empty table as a response.  I used a program called wget to retrieve 5,499 html pages, containing a total of 98,970 individual characters.

Here is the resulting information I managed to get 

\begin{itemize}
    \item 3,962 unique works consisting of 1 or more pages
    \item 110,652 unique characters with minimal bounding boxes annotated
    \item 920 unique Authors
    \item 19,976 unique scanned pages
\end{itemize}



\section{Downloading the source images}

\subsection{Organization of images on server}

Both character images and page live on the CADAL server in the following structure:


.
└── CalliSources
    ├── books
    │   ├── 06100004
    │   │   └── otiff
    │   │       ├── 00000009.tiff
    │   │       ├── 00000010.tiff
    │   │       └── .....
    │   ├── 06100005
    │   └── .....
    └── images
        └── chatacterimage
            ├── 06100004
            │   ├── 00000009(266,179,402,296).jpg
            │   ├── 00000010(206,251,358,346).jpg
            │   └── .....
            ├── 06100005
            └── .....




\subsection{Individual Character Jpegs}

On the server.  All files representing cropped characters are stored here:

http://www.cadal.zju.edu.cn/CalliSources/images/characterimage/


\subsection{Individual Page Images}





\section{Transforming data to a usefull format}

Unfortunately, while the HTML tables I received looked good when printed on a web-browser.  The source HTML was formatted in a non-standard way which made the included information difficult to extract automatically..  I built a Python Script using the BeautifulSoup library to extract all relavant information from the HTML table, and save this data in JSON.  Javascript Object Notation.  I chose JSON because it is becoming the de-facto standard for data sharing on the web.  Also python has a a JSON plugin which easily transforms python objects into JSON-readable format.  And I want to make my work as accessable as possible to fellow researchers.





\section{Aquiring Character and Page images}


\section{Building and Setting up a web-server}

\subsection{Why a webserver?}

I chose the webserver path because it provides the easiest and most natural.  Aditionally, my client is not tech-savy.  Unfortunately, I have very little experience programming for non-unix systems.  I do not wish to    Additionall, a web-server provides many advantages over a single-computer install.  It does not require a user to install any software, and permits many users to interact with the same product simultaniously.

\section{Building and running Django app}

Django provides the web developer with a great deal of flexibility.  The Django app itself is simply a regular Python program connected to a web-server through an interface.  The Python program receives commands from the user, then performs processing and sends the requested feedback, rendered as a web-page.  Importantly, the Django app has accesses to the computational resources of the server to preform significant work.  Since 

\section{sending stuff to the virtual server}


rsync -avr --progress newscan admin@167.114.93.26:~/





