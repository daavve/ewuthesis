\chapter{language choice}
\section{Python}
Python is a fully-featured high-level language. It has very mature image, machine learning, and scientific libraries.  I chose Python because of it's minimalist syntax and batteries-included nature.  Python's scripted nature lends itself to more rapid development than compiled languages such as Java and C. This is because changes in the Python codebase may be run immediately without waiting for a compiler to produce bytecode or machine code.\\
Python also works very well as a glue language to interface between different programs / libraries.
\subsection{scipy}
I used the Scientific Python package for much of my work.  SKimage provides a whole host of science modules out-of-the-box.
\section{not java}
Why not Java:  Java is a very good language. However, I found myself writing my own libraries for basic image processing / manipulation tasks.  Also Java does not integrate easily with C/C++ or Python.  Wile Java is very good at displaying images on screen, I found reading individual pixels slow, tedious, and difficult.
\subsection{if you must use java}
If you must use Java I recomend you download a program called ImageJ.  It is a software project provided by the department of Health and Human services or something.
\section{not C/C++}
Why no C/C++?:    A large base of image processing and scientific libraries exist for C++, and the language is very widely used.  However C++ is an enormous language, which I found challenging to learn. I found myself spending most of my time learning the vuageries of the language, with relatively little time doing active development.


\newpage
