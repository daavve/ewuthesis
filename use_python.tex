DIDN'T PAN OUT\\
\\
This work suffered from many false starts.\\
\\
Java Project:  I initially chose to build my project in Java. I did so because Java is the most popular language taught at EWU and I wanted to provide a software future students could build from to eventually create software package which historians might use. I built a prototype Graphical User Interface(GUI) which could open individual images. I also created an interactive segmentation engine, which broke a larger image into smaller parts recursively. The segmentation engine used a crude method of pixel averaging over the X and Y component of the image independently. The software was fairly modular and worked reliably, though it was unintuitave and somewhat inflexible. Ultimately I abandoned my Java software because the overwhelming majority of software available to work with images is in C++, Python, or Matlab. While it is possible to write Java adapters to interface with existing software from different languages. Doing so is both outside the scope of my project as well as too difficult and time-consuming to justify. Unfortunately, it is not possible to validate usefullness of my work scientifically without the ability to compare my results against existing work.


Gamera Plugin:  The Gamera project is the closest and most successfull implemintation of a Historical Document analysis framework I have yet seen. Written in Python2 and C++, It provides an environment by which the non-expert can build and tweak a custom workflow appropriate for a particular class of historical documents. The project also provides a plugin functionality which programmers can use to extend existing functionality by writing Python2 and C++ modules.

\newpage
