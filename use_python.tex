\chapter{Operating System and Language Choice}

\section{Operating System}
\subsection{Linux}
\subsubsection{ArchLinux}
All my work is perfomed on Linux.  I use the Arch Linux distribution because it is very up-to-date, and over 30,000 software package are easily installable through a centralized package manager.  Additionally, new software is continously added to the Arch User repository.  In fact, I specifically added python-theano, and python-mahotas packages to the AUR for anyone interested.
If you are unfamiliar with Linux I strongly reccomend Scientific Linux.  The linux distrobution comes prebuilt and set-up to perform science.  Most all the basic tools come pre-installed.
\subsection{Apple}
I recommend you go here to get started -> https://docs.python.org/3/using/mac.html
\subsection{Windows}
I recommend you check out winpython.  It comes with Python and Mahotas out-of-the-box.


\section{Language}
\subsection{Python}
Python is a fully-featured high-level language. It has very mature image, machine learning, and scientific libraries.  I chose Python because of it's minimalist syntax and batteries-included nature.  Python's scripted nature lends itself to more rapid development than compiled languages such as Java and C. This is because changes in the Python codebase may be run immediately without waiting for a compiler to produce bytecode or machine code.\\
Python also works very well as a glue language to interface between different programs / libraries.\\
Many variants of Python exist such as IronPython, RubyPython, JPython, Pypi and so on. I chose CPython, because CPython is the reference implementation of the Python language.  This means the CPython is considered the ``Official Language'' and most faithfully implements the Python Standard compared against other variants.
\subsubsection{Python2 or Python3}
I chose Python3 because it is the most up-to-date version of Python, and the one Python's designers want new software to be written in.
\subsubsection{Python and Sqlite}
Python classes can be used to hold data inside custom objects.  However, Python's class mechanism is not lightweight enough to use as a data-store of thousands of pieces of data.  Loading all characters into memory consumed over 3.2Gb of memory.  This strategy does not scale well, so I decided to use a more traditional database backed store.  I chose Sqlite because it is comes bundeled with Python and does not require the more complex setup of more heavyweight database engines such as MySQL.

\subsubsection{scipy and scipy-image}
I used the Scientific Python package for much of my work. SKimage of image analysis libraries out-of-the-box.  Python image and table data structures are written in C, with clear concise interfaces which permit direct editing by of in-memory data written in a variety of languages.

\subsection{not java}
Why not Java:  Java is a very good language. However, I found myself writing my own libraries for basic image processing / manipulation tasks.  Also Java does not integrate easily with C/C++ or Python.  The underlying image data is deeply burried inside the Java Virtual machine beneath many layers of abstraction. These digital images are stored in-memory in a format unique to Java, and must be translated to an intermediate representation such as BMP before an application written in a different language can interpret them.

\subsubsection{if you must use java}
If you must use Java I recomend you download a program called ImageJ. It is a software project provided by the department of Health and Human services or something.  The software is mostly used by biologists and health scientists. ImageJ has a mature pluggable design and a large repository of existing plugins.

\subsection{not C/C++}
A large base of image processing and scientific libraries exist for C++. C++ is the de-facto standard for image processing due to it's inherint speed and flexibility. Python has good enough integration with C/C++ that I can fairly easily integrate libraries written in C/C++ into my Python program.  In fact, most of the existing Cpython packages rely on C/C++ modules to perform the heavy lifting.  Additionally, since this is Academic software performance is not so important as rapid development.

\subsubsection{if you must use C++}
If you prefer to use C++ I strongly encourage you to checkout the OpenCV project. OpenCV is perhaps the most mature and capable computer vision library  available to-date.

\newpage
