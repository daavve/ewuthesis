\chapter{language choice}

\section{Python}
Python is a fully-featured high-level language. It has very mature image, machine learning, and scientific libraries.  I chose Python because of it's minimalist syntax and batteries-included nature.  Python's scripted nature lends itself to more rapid development than compiled languages such as Java and C. This is because changes in the Python codebase may be run immediately without waiting for a compiler to produce bytecode or machine code.\\
Python also works very well as a glue language to interface between different programs / libraries.\\
Many variants of Python exist such as IronPython, RubyPython, JPython, Pypi and so on. I chose CPython, because CPython is the reference implementation of the Python language.  This means the CPython is considered the ``Official Language'' and most faithfully implements the Python Standard compared against other variants.

\subsection{scipy and scipy-image}
I used the Scientific Python package for much of my work. SKimage of image analysis libraries out-of-the-box.  Python data structures are written in C, with clear concise interfaces which permit direct editing by of in-memory data for programs written in C, Fortran, C++, etc.

\section{not java}
Why not Java:  Java is a very good language. However, I found myself writing my own libraries for basic image processing / manipulation tasks.  Also Java does not integrate easily with C/C++ or Python.  Wile Java is very good at displaying images on screen, I found reading individual pixels slow, tedious, and difficult.  Java images are stored in-memory in a format unique to Java, and must be translated to an intermediate representation such as BMP before an application written in a different language can interpret them.

\subsection{if you must use java}
If you must use Java I recomend you download a program called ImageJ. It is a software project provided by the department of Health and Human services or something.

\section{not C/C++}
A large base of image processing and scientific libraries exist for C++.  C++ is the de-facto standard for image processing due to it's inherint speed and flexibility.  However C++ is an enormous language, which I found challenging to learn. I found myself spending most of my time learning the vuageries of the language, with relatively little time doing active development.

\subsection{if you must use C++}
If you prefer to use C++ I strongly encourage you to checkout the OpenCV project. It is perhaps the most mature and capable computer vision library  available to-date.

\newpage
