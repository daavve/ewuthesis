\chapter{building website}

\section{following django tutorial} % Note:  I used django version 1.9.1, should probably link to documentation of 1.9.1

django/html/intro/tutorial01.html

website: https://docs.djangoproject.com/en/1.9/contents/

I did the following:


python -c "import django; print(django.get_version())"
1.9.1

Notes:  Project holds the basic settings and stuff for a website,  app provides an application for a website.


[dave@bigArch kate]$ django-admin startproject ewucal
[dave@bigArch kate]$ tree
.
└── ewucal
    ├── ewucal
    │   ├── __init__.py
    │   ├── settings.py
    │   ├── urls.py
    │   └── wsgi.py
    └── manage.py

    Now, we make our first app in the project.
    
[dave@bigArch ewucal]$ python manage.py startapp calligraphy
[dave@bigArch ewucal]$ tree
.
├── calligraphy
│   ├── admin.py
│   ├── apps.py
│   ├── __init__.py
│   ├── migrations
│   │   └── __init__.py
│   ├── models.py
│   ├── tests.py
│   └── views.py
├── ewucal
│   ├── __init__.py
│   ├── __pycache__
│   │   ├── __init__.cpython-35.pyc
│   │   └── settings.cpython-35.pyc
│   ├── settings.py
│   ├── urls.py
│   └── wsgi.py
└── manage.py

we need to create a file inside calligraphy called urls.py
from django.conf.urls import url
from . import views

urlpatterns = [
    url(r'^\$', views.index, name='index')
]


now edit views.py inside calligraphy
from django.http import HttpResponse

def index(request):
    return HttpResponse("Calligraphy App.")
   
now edir urls.py inside ewucal to point to our new Calligraphy app:
from django.conf.urls import url, include
from django.contrib import admin

urlpatterns = [
    url(r'^calligraphy/', include('calligraphy.urls')),
    url(r'^admin/', admin.site.urls),
]


\section{connecting to database}


now, that the database is connected:  Run:  \$ python manage.py migrate

This populates the Database with the Installed Apps.

[dave@bigArch ewucal]\$ python manage.py migrate
Operations to perform:
  Apply all migrations: admin, contenttypes, auth, sessions
Running migrations:
  Rendering model states... DONE
  Applying contenttypes.0001_initial... OK
  Applying auth.0001_initial... OK
  Applying admin.0001_initial... OK
  Applying admin.0002_logentry_remove_auto_add... OK
  Applying contenttypes.0002_remove_content_type_name... OK
  Applying auth.0002_alter_permission_name_max_length... OK
  Applying auth.0003_alter_user_email_max_length... OK
  Applying auth.0004_alter_user_username_opts... OK
  Applying auth.0005_alter_user_last_login_null... OK
  Applying auth.0006_require_contenttypes_0002... OK
  Applying auth.0007_alter_validators_add_error_messages... OK
  Applying sessions.0001_initial... OK

  
  Note:  All this stuff comes with the Application deffinition, I probably don't need most of it.
  
  
  
  
  
  
\section{building a schema}

Django calls this the Object relational model.

our files organizatuin

├── CalliSources
│   ├── books
│   │   ├── 06100004
│   │   │   ├── 00000009.jpg
│   │   │   ├── 00000010.jpg
│   │   │   ├── 00000011.jpg
........

│   │   ├── 06100006
│   │   │   ├── 00000034.jpg
│   │   │   ├── 00000035.jpg
.....
│   └── characterimage
│       ├── 06100004
│       │   ├── 00000009(266,179,402,296).jpg
│       │   ├── 00000009(267,76,343,175).jpg
│       │   ├── 00000009(284,313,346,406).jpg
│       │   ├── 00000009(387,492,471,654).jpg
│       │   ├── 00000009(388,75,473,274).jpg
│       │   ├── 00000009(401,325,463,391).jpg
......
│       ├── 06100006
│       │   ├── 00000034(125,595,238,668).jpg
│       │   ├── 00000034(129,778,225,848).jpg
│       │   ├── 00000034(131,680,223,766).jpg




our JSON Data.......

    {
        "chi_author": "王羲之",
        "chi_mark": "夫",
        "chi_work": "兰亭序",
        "page_id": "00000055",
        "work_id": "06100007",
        "xy_coordinates": [
            "568",
            "438",
            "634",
            "507"
        ]
    }
    
    
    
    Since we have 3 significant families of objects.  Books, Pages, & Characters.  Probably makes sense to do one table for each.
    
Create the following models.py

from django.db import models


class Book(models.Model):
    book_id = models.IntegerField(primary_key=True)
    book_title = models.CharField(max_length=16)
    book_author = models.CharField(max_length=16)
    book_transcriber = models.CharField(max_length=16,blank=True)
    book_notes = models.TextField(blank=True)


class Page(models.Model):
    page_number = models.IntegerField()
    book_parent = models.ForeignKey(Book, on_delete=models.CASCADE)
    book_notes = models.TextField(blank=True)


class Character(models.Model):
    page_parent = models.ForeignKey(Page, on_delete=models.CASCADE)
    char_mark = models.CharField(max_length=1, blank=True)
    x1 = models.IntegerField()
    y1 = models.IntegerField()
    x2 = models.IntegerField()
    y2 = models.IntegerField()
    char_notes = models.TextField(blank=True)
    
    
insert ``'calligraphy.apps.CalligraphyConfig','' into INSTALLED_APPS array

python manage.py makemigrations calligraphy
Migrations for 'calligraphy':
  0001_initial.py:
    - Create model Book
    - Create model Character
    - Create model Page
    - Add field page_parent to character
    
Now the following gives the output DDL used to create the new Schema

[dave@bigArch ewucal]\$ python manage.py sqlmigrate calligraphy 0001
BEGIN;
--
-- Create model Book
--
CREATE TABLE "calligraphy_book" ("book_id" integer NOT NULL PRIMARY KEY, "book_title" varchar(16) NOT NULL, "book_author" varchar(16) NOT NULL, "book_transcriber" varchar(16) NOT NULL, "book_notes" text NOT NULL);
--
-- Create model Character
--
CREATE TABLE "calligraphy_character" ("id" serial NOT NULL PRIMARY KEY, "char_mark" varchar(1) NOT NULL, "x1" integer NOT NULL, "y1" integer NOT NULL, "x2" integer NOT NULL, "y2" integer NOT NULL, "char_notes" text NOT NULL);
--
-- Create model Page
--
CREATE TABLE "calligraphy_page" ("id" serial NOT NULL PRIMARY KEY, "page_number" integer NOT NULL, "book_notes" text NOT NULL, "book_parent_id" integer NOT NULL);
--
-- Add field page_parent to character
--
ALTER TABLE "calligraphy_character" ADD COLUMN "page_parent_id" integer NOT NULL;
ALTER TABLE "calligraphy_character" ALTER COLUMN "page_parent_id" DROP DEFAULT;
ALTER TABLE "calligraphy_page" ADD CONSTRAINT "calligraphy_book_parent_id_afbd29d9_fk_calligraphy_book_book_id" FOREIGN KEY ("book_parent_id") REFERENCES "calligraphy_book" ("book_id") DEFERRABLE INITIALLY DEFERRED;
CREATE INDEX "calligraphy_page_0b2bd891" ON "calligraphy_page" ("book_parent_id");
CREATE INDEX "calligraphy_character_577b43c4" ON "calligraphy_character" ("page_parent_id");
ALTER TABLE "calligraphy_character" ADD CONSTRAINT "calligraphy_char_page_parent_id_ec7668c6_fk_calligraphy_page_id" FOREIGN KEY ("page_parent_id") REFERENCES "calligraphy_page" ("id") DEFERRABLE INITIALLY DEFERRED;

COMMIT;

[dave@bigArch ewucal]\$ python manage.py migrate
Operations to perform:
  Apply all migrations: calligraphy, sessions, admin, auth, contenttypes
Running migrations:
  Rendering model states... DONE
  Applying calligraphy.0001_initial... OK

See image to view graphical discription of schema.

workcycle is:  
Change your models (in models.py).
Run python manage.py makemigrations to create migrations for those changes
Run python manage.py migrate to apply those changes to the database.



  
  
Needed to add:     jsonfile = open(filename, "r", encoding="utf-8") the encoding=``utf-8'' because the server I have is set up for ASCII

Needed to add the following above the.models file:

def import_json(apps, schema_editor):
    Booki = apps.get_model("calligraphy", "Book")
    Pagei = apps.get_model("calligraphy", "Page")
    Characteri = apps.get_model("calligraphy", "Character")

    books = []

    chars = readjson("dump.json")
    for char in chars:
        inserttobook(char, books)

    for book in books:
        bk = Booki(book_id=book.bid,
                  book_title=book.title,
                  book_author=book.author)
        bk.save()
        for pag in book.pages:
            pg = Pagei(page_number=pag.number,
                      book_parent=bk)
            pg.save();
            for cher in pag.characters:
                if cher.mark is None:
                    chmark = ""
                else:
                    chmark = cher.mark
                ch = Characteri(char_mark=chmark,
                               x1=cher.x1,
                               y1=cher.y1,
                               x2=cher.x2,
                               y2=cher.y2,
                               page_parent=pg)
                ch.save()
                
                
Then needed to run :

\$ createdb calligraphy

Then:

[dave@bigArch ewucal]\$ python manage.py migrate
Operations to perform:
  Apply all migrations: contenttypes, sessions, calligraphy, admin, auth
Running migrations:
  Rendering model states... DONE
  Applying contenttypes.0001_initial... OK
  Applying auth.0001_initial... OK
  Applying admin.0001_initial... OK
  Applying admin.0002_logentry_remove_auto_add... OK
  Applying contenttypes.0002_remove_content_type_name... OK
  Applying auth.0002_alter_permission_name_max_length... OK
  Applying auth.0003_alter_user_email_max_length... OK
  Applying auth.0004_alter_user_username_opts... OK
  Applying auth.0005_alter_user_last_login_null... OK
  Applying auth.0006_require_contenttypes_0002... OK
  Applying auth.0007_alter_validators_add_error_messages... OK
  Applying calligraphy.0001_initial... OK
  Applying sessions.0001_initial... OK

  
Actual Migraiton code:

class Migration(migrations.Migration):

    initial = True

    dependencies = [
    ]

    operations = [
        migrations.CreateModel(
            name='Book',
            fields=[
                ('book_id', models.IntegerField(primary_key=True, serialize=False)),
                ('book_title', models.CharField(max_length=16)),
                ('book_author', models.CharField(max_length=16)),
                ('book_transcriber', models.CharField(blank=True, max_length=16)),
                ('book_notes', models.TextField(blank=True)),
            ],
        ),
        migrations.CreateModel(
            name='Character',
            fields=[
                ('id', models.AutoField(auto_created=True, primary_key=True, serialize=False, verbose_name='ID')),
                ('char_mark', models.CharField(blank=True, max_length=4)),
                ('x1', models.IntegerField()),
                ('y1', models.IntegerField()),
                ('x2', models.IntegerField()),
                ('y2', models.IntegerField()),
                ('char_notes', models.TextField(blank=True)),
            ],
        ),
        migrations.CreateModel(
            name='Page',
            fields=[
                ('id', models.AutoField(auto_created=True, primary_key=True, serialize=False, verbose_name='ID')),
                ('page_number', models.IntegerField()),
                ('book_notes', models.TextField(blank=True)),
                ('book_parent', models.ForeignKey(on_delete=django.db.models.deletion.CASCADE, to='calligraphy.Book')),
            ],
        ),
        migrations.AddField(
            model_name='character',
            name='page_parent',
            field=models.ForeignKey(on_delete=django.db.models.deletion.CASCADE, to='calligraphy.Page'),
        ),
        migrations.RunPython(import_json)
    ]
    
    
    
    [dave@bigArch ewucal]\$ python manage.py createsuperuser
Username (leave blank to use 'dave'): 
Email address: davidm@eagles.ewu.edu
Password: 
Password (again): 
Superuser created successfully.






\section{admin Stuff}


Getting yaourt set up:



wget https://aur.archlinux.org/cgit/aur.git/snapshot/package-query.tar.gz

gzip -d package-query.tar.gz

tar -xf package-query.tar

cd package-query
makepkg
