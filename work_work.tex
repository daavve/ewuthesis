\chapter{work-work}



\subsubsection{Python and Databases}
Python classes can be used to hold data inside custom objects.  However, Python's class mechanism is not particularly lightweight.  My data-holding object consumes arount 1.5Mb of memory.  This might not seem like much, but consider that each class holds 9 strings.  4 of these strings are guarenteed to be 4 characters or less, the other 5 are 8 characters or less.  I am using the UTF-8 Unicode standard, which means each character uses 8 bits.  So, (4 strings * 4 characters + 5 strings * 8 charactyers) * 8 bits = 288 bits.  Considering That there are 1024 bytes in a Kb and 1024 Kb in a MB, then I paying over a 5 thousand-fold cost in increased memory use for the convience of using Python Objects to hold hold my data.

I eventually decided to use a CouchDB database, because the entries into a couchDB database are JSON, and I don't have to bother creating a schema or joins or the like....


\subsubsection{Why use a database at all}

Django expects a database to hold relevant.

\subsubsection{Database Choice}

I initially chose CouchDB because the underlying entries into the couchDB database are JSON, and since I already had JSON data for my characters.

Unfortunately, I found CouchDB not a good choice, because Django expects to connect to a SQL-like database, and CouchDB is more of a Key-Value store.  I did get Django to successfully interface with CouchDB, but the resulting construct was extremely slow.



\subsubsection{Python and Django}

But I found out that Django is the widely used and among the most mature Python frameworks.  Django uses something called an Object relational layer to interface with Databases.  Unfortunately, Django pretty much expects to connect to a SQL-Like database.  It is possible to connect 

I did get Django to work with CouchDB


\subsection{SQL Schema}



\subsection{Python and Classes}

python classes can be extended using the following syntax:
    class Extendedclass(Classtobeextended):

\subsection{Python and JSON}
JSON(Java Script Object Notation) is a popular data-interchange format used to


\section{general issues with Python}

\subsection{Garbage Collection}

Programs such as Python perform something called Garbage collection.  This process is the periodic scrubbing of memory for objects which are no longer used.\\
I am occustomed to writing in Java.  In Java, the JVM(Java Virtual Machine) Keeps very close track of all objects existing in memory as well as how objects depend on each other.  Whenever lots of memory becomes used the JVM preiodically pauses execution in order to clean unused objects.\\
I feel that in Python, memory management must work differently.  As Python feels strongly inclined to consume additional memory, and not so strongly inclined to free memory used(at least not automatically).  I also feel that object management must operate differently in Python, because light objects spawned within heavier objects, and then passed out seem to ``inherit'' a good deal of bloat from the object that orchistrated it's creation.



\section{webserver}

\subsection{The world wide web}

For our purposes. the world wide web is defined as a vast communication network allowing computers worldwide the ability to send and receive information from each other.  The computers on this network are broadly defined as either server or client.

\subsection{anatomy of a client}

A client machine is a computer which people use to access the world wide web.  The client computer contains a program called a browser.  A browser is capable of sending 

\subsection{anatomy a server}

A is simply a computer which exists on the internet that is capable of sending information on request to users.  The organization and context of this data is what constitutes a website.  Webservers send two types of content to a user.  Static and dynamic.

\subsection{The web browser}

A web browser is a software program installed on the user's computer, which (when ran) permits the user a means of interacting with the world-wide web.  


\subsubsection{the web page}

A web page is an individual document composed of markup text which instructs the web browser on how to construct a web page

\chapter{Markup Text}

Fundamentel to the internet is the concept of markup text.  When a browser downloads a 

Hypertext markup Language, or HTML is the basic building block of a website.  The HTML defines the content of a website.  A web page 

\chapter{css}

Cascading Style Sheets describe to the browser how the images and text of a website should be organized.  

\chapter{javascript}

Javascript is a programming language which most web-browsers are capable of running.  From the context of our project.  Javascript provides the interactivity of the website.  It is especially usefull because thee scripts respond programmatically to user inputs such as selecting, and dragging and dropping.  






\subsubsection{static content}

static content consists of documents necessary for the generation of a website that do not change fequently or at all.  The calligraphy documents we 

\paragraph{images}

Images are prehaps the most well-known form of static content.  They consists

\paragraph{javascript libraries}

Javascript libraries exist to simplify and steamline the process of writing Javascript for a website.  The javascript is 

\subsubsection{dynamic content}


\subsection{Django}

Python documentation recommended I use this simple server such as gunicorn, or uwsgi.  I chose uwsgi, because it's more popular, and probably faster, but I think gunicorn would work fine too.


\subsection{wsgi}





\subsection{javascript libraries}

\subsubsection{jquery}

\paragraph{jquery-ui}

Is a jquery package made by the jquery foundation.  It is used to provide users with 

\paragrarph{jrac}

JRAC(Jquery Resize and Crop) is a a jquery package that uses jquery-ui to provide the user an interactive window 

\subsection{if you want to enable image uploading by users}

https://pixabay.com/en/blog/posts/direct-image-uploads-in-tinymce-4-42/

\subsection{it's sooo obviouis}


